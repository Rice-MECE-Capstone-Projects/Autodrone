\subsection{YOLOv8}\label{subsection:C}
Object detection is a crucial aspect of autonomous drone operation. While there are many deep learning models that can be used for object detection, Convolutional Neural Networks (CNNs) are the most researched method. However, these models mostly use a two-stage pipeline which may not be efficient for autonomous drone purposes. In contrast, one-stage models such as YOLO (You Only Look Once) \cite{2c1} model is a state-of-the-art real-time object detection algorithm, known for its high speed and accuracy. YOLO processes the input image by dividing it into a grid and predicting a set of bounding boxes, objectness scores, and class probabilities for each grid cell. These predictions are combined and subjected to non-maximum suppression to remove overlapping boxes, generating the final object detection. During this semester, our team underwent a significant transition from YOLOv5\cite{2c2} to the latest iteration, YOLOv8\cite{2c3}, in our pursuit of advancing object detection and segmentation capabilities. The model is built on a foundation of extensive pre-training on millions of images and places a particular emphasis on its zero-shot learning capabilities.

In our AutoDrone Re-ID project, our primary focus is on the detection of people using a webcam mounted on the drone. The YOLOv8 model assumes a crucial role in identifying individuals within the camera frame, thereby enhancing the overall efficiency and safety of the drone's operation. The YOLOv8 model is implemented using the PyTorch framework, offering a diverse selection of pre-trained models, each striking a unique balance between speed and accuracy. Our team opted for the YOLOv8s model, a more compact version with a parameter count of 11.2 millions. This strategic decision ensures optimal performance within the limitations of the drone's hardware.

In the context of our AutoDrone project, the YOLOv8 model has successfully demonstrated several key functionalities. Utilizing the YOLOv8 model, our implementation enables precise people counting in a given scene over a specified period. This capability facilitates the comprehensive analysis of crowd dynamics, providing valuable insights for operational decision-making. Furthermore, in alignment with the global emphasis on safety measures, our YOLOv8 model has been enhanced to incorporate mask detection capabilities by trained with a random sample of MaskedFace-Net dataset\cite{2c4}. It can accurately identify individuals, distinguishing between those wearing masks and those without, thereby adding an extra layer of safety in public spaces.