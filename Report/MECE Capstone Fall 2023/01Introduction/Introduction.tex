\section{Introduction}

Recent advancements in person Re-ID are classified into four main categories based on metric and representation learning: deep metric learning, local feature learning, generative adversarial learning, and sequence feature learning. Each of these categories is further subdivided based on methodologies and motivations, which helps in understanding their unique advantages and limitations. These classifications signify the depth and breadth of current research in person Re-ID, highlighting how each approach contributes to overcoming the challenges posed by varied camera views, poses, illumination, and resolution\cite{1a1}.

The integration of autonomous drones in person Re-ID presents a novel approach to addressing the limitations of current methods, especially in complex environments like crowded spaces with non-uniform lighting. The project's future focus involves refining drone capabilities and AI algorithms to enhance accuracy and adaptability. This direction aligns with the ongoing research trends in person Re-ID, where the emphasis is on developing more robust, efficient, and adaptable systems for real-world applications.

Central to this project is the integration of the digital twin concept, a virtual replica of the physical drone. The use of Software in the Loop (SITL) simulation is instrumental in this project, which ensures robust autonomous decision-making. ROS, a comprehensive middleware framework for robot software development, facilitates efficient integration of computational resources and algorithms. The combination of these advanced technologies, SITL simulation and ROS, enhances the drone's performance and flexibility in dynamic environments.